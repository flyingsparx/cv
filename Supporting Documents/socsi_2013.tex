\documentclass[11pt,a4paper]{article}
\usepackage[utf8]{inputenc}
\usepackage{amsmath}
\usepackage{amsfonts}
\usepackage{amssymb}
\usepackage[margin=0.87in]{geometry}
\title{Will M. Webberley - Application Supporting Information}
\date{}
\begin{document}
\maketitle
Please find information below to support my claims to the position's required essential criteria.\\

\textbf{Evidence of written and verbal skills}\\
Throughout my PhD and undergraduate degree I have produced many instances of high quality written work, including the publication of academic research papers in proceedings and funding proposals. I have also demonstrated excellent verbal skills through presentation-style tutorials and lecture-type sessions to students, informal talks, and formal research talks in international conference workshops.\\
Over the past few years I have been awarded the School of Computer Science \& Informatics' ``best presentation'' award for research students and I have won various poster day prizes on multiple occasions.\\

\textbf{Communicating complex ideas}\\
The field of computer science is among those that sometimes require understanding hypothetically conceptual (and non-practical) situations, and when teaching these ideas to less-experienced undergraduate students I have had to find methods for conveying the information in an effective and comprehensive fashion.\\
In addition, parts of my PhD research are very socially oriented, furthering my experiences in communicating research ideas to interdisciplinary audiences, such as at the recent SCA conference.\\

\textbf{Course planning, development, teaching \& assessment in social methods}\\
During my PhD I have been responsible for the teaching and assessment of social-related student projects, including those studying and manipulating digital social data. I have also provided input on the direction of courses in terms of the tools used and the provision of support sessions.\\
My responsibilities with teaching also extend to the assessment of student coursework in a range of formats, and being responsible for the planning, supervision, and assessment of year-long group projects.\\

\textbf{Interest in the analysis of social science data}\\
My PhD covered many aspects of social research, including the analysis of Twitter data, the use of validation tools such as Mechanical Turk and Crowdflower, and machine learning and statistical techniques, such as Bayesian networks and logistic regressions. In addition to this work, I have been interested in other aspects of social science research, including the organisation of and an application for funding (which was successful in receiving £850) for an interdisciplinary `hackathon' between the Schools of Social Science and Computer Science \& Informatics.\\
In my spare time I have carried out individual development projects, some of which involve the use of social sources, including Android apps on Google Play.\\
I am also currently carrying out some collaborative and interdisciplinary research, which links the in-depth term and locational data in Twitter to various social-related behaviours. As part of my teaching, I have elected to supervise projects involving the production of social media visualisation toolkits, which involved correspondence with representatives from COSMOS.\\

\textbf{Knowledge of skills and techniques for quantitative analysis}\\
I am aware of and have experience with various techniques used from data collection, storage, and through to the quantitative analysis of the data. I enjoy discovering new features in data in order to support findings by providing further rigidity to results.\\
I have knowledge of how to find the appropriate methods for analysing data in different ways, am experienced in using Python for data analysis and have some experience with statistical tools, such as R and Matlab, and validation and crowd-sourcing services, including Mechanical Turk and Crowdflower.\\

\textbf{Demonstration of the understanding of primary and secondary research}\\
I have been involved in many instances of data analysis as part of primary and secondary research. The majority of my PhD has involved using data I've directly collected from sources, such as Twitter. In addition, I've been involved with the collection from media such as Last.fm and IMDB for use in research projects. I also have knowledge of and experience with various tools and practices behind uses of APIs for collecting data and for the handling and storage of high data volumes.\\
Challenges I've experienced with this type of research include handling service rate-limit policies and unstructured data, and I have had to find ways of organising data collection to cope with these.\\
Examples of secondary research include projects involving the use of police and government data on crime, crime outcome, food hygiene, and pre-collected Twitter data in a variety of research projects and during my PhD.\\

\textbf{Development of research and funding proposals \& collaborative work}\\
Last year I co-authored an application for funding, which was successful in achieving £850, for an interdisciplinary `hackathon' event between the schools of Social Science and Computer Science \& Informatics. It was successful and several small research projects involving social data emerged from the short-term collaborations, including those using healthcare, crime, and behavioural data.\\
Further collaborative work includes contributions towards the TaRDiS project, which relates social media to real-life social behaviour in South Wales and London, and contribution to the EU Recognition project, including to the final deliverables, which was given an ``excellent'' rating.\\

\textbf{Publication potential of research findings}\\
During my PhD I have periodically published work of interest in relevant conference workshop papers, which receive interesting citations, and I have presented these in Italy and Germany in addition to posters demonstrating my research in more local events. I am currently planning a journal paper based on the final stages and interesting findings from my PhD. Some of my work has also featured in contribution to other research projects in the School of Computer Science \& Informatics.\\

\textbf{Experience in undergraduate and postgraduate teaching}\\
Throughout the past three years of my PhD I have been responsible for carrying out teaching duties for the undergraduate and MSc students in the School of Computer Science \& Informatics. These have included running tutorials, practical lab sessions, and lecture style-sessions several times a week for groups of between 40 and up to 140 students at a time.\\
I have also been the supervisor in a series of year-long undergraduate group projects, providing assistance and being responsible for several groups, each containing 6-10 students.\\
The teaching has included the assessment of the students in a variety of formats, including periodic coursework assessment, presentations, reports, and practical demos.\\
I have given tutorial-style talks to research students and research groups in the School of Computer Science \& Informatics on tools such as the WEKA machine learning toolkit, \LaTeX, and platforms including Node.js and Flask.\\
Courses I've attended, run by the UGC, include those on assessment and demonstrating student learning in the sciences.\\

\textbf{Higher-level degree status}\\
I am currently approaching the final stages of writing up my PhD thesis, which I plan to submit by Spring 2014.

\end{document}
