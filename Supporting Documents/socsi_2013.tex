\documentclass[10pt,a4paper]{article}
\usepackage[utf8]{inputenc}
\usepackage{amsmath}
\usepackage{amsfonts}
\usepackage{amssymb}
\usepackage[margin=1in]{geometry}
\title{Application:	 Supporting Information}
\author{Will M. Webberley}
\date{}
\begin{document}
\maketitle

Please find information below to support my claims to the position's required essential criteria.

\section*{Evidence of written and verbal skills}
Throughout my PhD and undergraduate degree, I have been required to produce many instances of high quality written work, including the production of academic research papers and funding proposals. I have also demonstrated excellent verbal skills through presentation-style tutorials and lecture-style sessions to students, informal talks, formal research talks in international conference workshops, and have been awarded the School of Computer Science \& Informatics' ``best presentation'' for research students several times.

\section*{Communicating complex ideas}
The field of computer science is among those that sometimes require understanding hypothetically conceptual (and non-practical) situations, and when teaching these ideas to less-experienced undergraduate students I have had to find methods for conveying the information in an effective and comprehensive fashion.\\

\section*{Course planning, development, teaching \& assessment in social methods}
During my PhD I have been responsible for the teaching and assessment of social-related student projects, including those studying and manipulating digital social data. I have also provided input on the direction of courses in terms of the tools used and the provision of support sessions.\\
My responsibilities with teaching also extend to the assessment of student coursework in a range of formats, and being responsible for the planning, supervision, and assessment of year-long group projects.

\section*{Interest in the analysis of social science data}
In addition to my PhD, I have been interested in other aspects of social science research, including the organisation of an interdisciplinary `hackathon' between the Schools of Social Science and Computer Science \& Informatics and, more recently, collaboration with the School of Social Science on relating social media cues to real-life social behaviours.\\
I am also currently doing some part-time research linking the in-depth term and locational data of Twitter to various social-related behaviour. As part of my teaching, I have elected to supervise projects involving the production of social media visualisation toolkits, which involved correspondence with representatives from COSMOS.

\section*{Knowledge of skills and techniques for quantitative analysis}
I am aware of and have experience with various techniques used from data collection, storage, and through to the quantitative analysis of the data. I enjoy discovering new features in data in order to support findings by providing further rigidity to results.\\
I have knowledge of how to find the appropriate methods for analysing data in different ways, am experienced in using Python for data analysis and have some experience with tools such as R and Matlab.

\section*{Demonstration of the understanding of primary and secondary research}
I have been involved in many instances of data analysis as part of primary and secondary research. The majority of my PhD has involved using data I've directly collected from sources, such as Twitter. In addition, I've been involved with the collection from media such as Last.fm and IMDB for use in research projects. I also have knowledge of and experience with various tools and practices behind uses of APIs for collecting data and for the handling and storage of high data volumes.\\
Examples of seondary research include projects involving the use of police and government data on crime, crime outcome, food hygiene, and pre-collected Twitter data in a variety of research projects and during my PhD.

\section*{Development of research and funding proposals \& collaborative work}
Last year I co-authored an application for funding for an interdisciplinary `hackathon' event between the schools of Social Science and Computer Science \& Informatics, which was successful, and several short research projects involving social data emerged from the short-term collaboriations.\\
Further collaborative work includes contributions towards the TaRDiS project, which relates social media to real-life social behaviour in South Wales and London, and some contribution to the EU Recognition project. 

\section*{Publication potential of research findings}
During my PhD I have periodically published work of interest in relevant conference workshop papers and have presented these in Italy and Germany in addition to posters demonstrating my research in more local events. Some of my work has been in contribution to other research projects in the School of Computer Science \& Informatics.

\section*{Experience of undergraduate and postgraduate teaching}
Throughout the past three years of my PhD I have been responsible for carrying out teaching duties for the undergraduate and MSc students in the School of Computer Science \& Informatics. These have involved running tutorials, practical lab sessions, lecture style-sessions, and so on, several times a week for groups of up to 140 students at a time.\\
I have also been the supervisor in a series of year-long undergraduate group projects, providing assistance and being responsible for several groups, each containing 6-10 students.\\
The teaching has included the assessment of the students in a variety of formats, including periodic coursework assessment, presentations, reports, practical demos, and so on.

I have attended courses, run by the UGC, on assessment and demonstrating student learning in the sciences.

\section*{Higher-level degree status}
I am currently approaching the final stages of writing up my PhD thesis, which I plan to submit and complete by Spring 2014.

\end{document}
