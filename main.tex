%----------------------------------------------------------------------------------------
%	PACKAGES AND OTHER DOCUMENT CONFIGURATIONS
%----------------------------------------------------------------------------------------

\documentclass[11pt,a4paper,sans]{moderncv} % Font sizes: 10, 11, or 12; paper sizes: a4paper, letterpaper, a5paper, legalpaper, executivepaper or landscape; font families: sans or roman

\moderncvstyle{classic} % CV theme - options include: 'casual' (default), 'classic', 'oldstyle' and 'banking'
\moderncvcolor{green} % CV color - options include: 'blue' (default), 'orange', 'green', 'red', 'purple', 'grey' and 'black'

\usepackage[scale=0.75]{geometry} % Reduce document margins
%\setlength{\hintscolumnwidth}{3cm} % Uncomment to change the width of the dates column
%\setlength{\makecvtitlenamewidth}{10cm} % For the 'classic' style, uncomment to adjust the width of the space allocated to your name


%----------------------------------------------------------------------------------------
%	NAME AND CONTACT INFORMATION SECTION
%----------------------------------------------------------------------------------------

\firstname{Will} % Your first name
\familyname{Webberley} % Your last name

% All information in this block is optional, comment out any lines you don't need
\title{Curriculum Vitae}
\address{School of Computer Science \& Informatics}{Cardiff University, Cardiff. CF24 3AA}
\mobile{07787 567865}
\social[twitter]{@flyingSparx}
\social[github]{flyingsparx} 
\homepage{flyingsparx.net}
\email{W.M.Webberley@cs.cardiff.ac.uk}


%----------------------------------------------------------------------------------------

\begin{document}

\makecvtitle % Print the CV title

%----------------------------------------------------------------------------------------
%	EDUCATION SECTION
%----------------------------------------------------------------------------------------

\section{Education}

	\cventry{2010--2014\\(expected)}{PhD Computer Science}{Cardiff University}{}{\newline{}"\textit{Inferring Interestingness in Online Social Networks}"}{Exploring information propagation in online social networks. In particular, the use of Bayesian machine learning methodologies in the prediction of Tweet popularity in Twitter as a basis for the inference and ranking of interesting Tweets.\\
		}
	\cventry{2007--2010}{BSc Computer Science}{Cardiff University}{}{}{First Class Honours}
	\cventry{2006--2007}{A-Levels}{King's School, Worcester}{}{}{
		\textit{To A2-level:} Mathematics, Biology, Chemistry \\
		\textit{To AS-level:} Physics
}

	
%----------------------------------------------------------------------------------------
%	WORK EXPERIENCE SECTION
%----------------------------------------------------------------------------------------

\section{Experience \& Employment}

	\cventry{2010--Present}{Teaching assistant and student supervisor}{School of Computer Science \& Informatics, Cardiff University}{UK}{}{
	Many hours a week of teaching of undergraduate and masters students on a wide range of topics as a module tutor, year-long group supervisor, practical lab leader, and in lecture-style sessions. Also responsible for the assessment (including feedback) of student work in the form of coursework, practical demos, project reports, and presentations.
\newline{}\textbf{Examples of subjects:} Python, \textsc{Java}, web application programming, computer architecture, mobile communications and meta heuristics, algorithms and data structures, communication \& professional skills, assembly language
\newline{}\textbf{Examples of supervised projects:} Social media visualisation toolkits, coursework assistance tools, user-contributed geo-information sharing mobile app design}

	\cventry{2010}{Abercrombie \& Fitch}{Hollister Impact}{}{}{Responsible for organising the shop floor, receiving and oragnising shipments, keeping track of stock and replenishment. Skills gained included communication across teams in a multi-discipline environment, precise organisation and being accurate with stock.}

	\cventry{2009}{Pulse Nightclub}{Bar staff}{}{}{Bar work as part of a vibrant and outgoing team. Direct contact with customers helped improve communication and teamwork skills, confidence, quick-thinking in a very fast-paced environment and over prolonged lengths of time. Responsible also for the supervision of new members of staff.}

%----------------------------------------------------------------------------------------
%	PUBLICATIONS SECTION
%----------------------------------------------------------------------------------------

\section{Peer-Reviewed Publications}
	\cventry{Sept. 2013}{Inferring the Interesting Tweets in Your Network}
		{\newline Will Webberley, Stuart M Allen, Roger M Whitaker}
		{\newline in Workshop on Analyzing Social Media for the Benefit of Society (SOCIETY2.0),\newline{}3\textsuperscript{rd} International Conference on Social Computing and its Applications (SCA)}{}{\textit{At:} Karlsruhe, Germany / \textit{Reviewed:} Blind peer-reviewed / \textit{Publisher:} IEEE}
	
	\cventry{Sept. 2011}{Retweeting: A Study of Message-Forwarding in Twitter}
		{\newline Will Webberley, Stuart M Allen, Roger M Whitaker}
		{\newline in Workshop on Mobile and Online Social Networks (MOSN'11),\newline{} 5\textsuperscript{th} International Conference on Network and System Security (NSS)}{}{\textit{At:} Milan, Italy / \textit{Reviewed:} Double-blind peer-reviewed / \textit{Publisher:} IEEE}



%----------------------------------------------------------------------------------------
%	COMMUNITY SECTION
%----------------------------------------------------------------------------------------


\section{Community Involvement \& Contribution}
	\cvitem{Articles}{
		\begin{itemize}
			\item "Direct-to-S3 File Uploads in Python", \textit{Heroku 2013} \begin{itemize}
				\item Invited development article for Heroku's \textit{Dev Center}
			\end{itemize}
		\end{itemize}
	}
	
	\cvitem{Talks}{
		\begin{itemize}
			\item \textbf{Informal}\begin{itemize}		
				\item "The Flask Microframework", \textit{2013}
				\item "Node.js", \textit{2012}
				\item "Introduction to \LaTeX", \textit{2012}
				\item "Developing Google Chrome Extensions", \textit{2011}
			\end{itemize}
			\item \textbf{Formal / academic}\begin{itemize}
				\item "Inferring the Interesting Tweets in Your Network", \textit{SOCIETY2.0, Karlsruhe, Germany, 2013}
				\item "Finishing a PhD in Under 4 Years", \textit{COMSC Research Retreat, 2013}
				\item "Inferring interesting Tweets from your Network", \textit{COMSC Informatics Group Seminar, 2013}
				\item "Modelling the behaviour of retweets", \textit{COMSC Research Retreat, 2012}
				\item "Retweeting: A Study of Message-Forwarding in Twitter", \textit{MOSN'11, Milan, Italy, 2011}
				\item "Getting relevant information to the right people in social networks", \textit{COMSC Research Retreat, 2011} 
			\end{itemize}
		\end{itemize}
	}

	\cvitem{Events}{
		\begin{itemize}
			\item \textbf{Organised} \begin{itemize}
				\item FTS seminar series, \textit{School of Computer Science \& Informatics 2010-2013}
				\item DigiSocial Hackathon, \textit{2012}			
			\end{itemize}
			\item \textbf{Taken part in} \begin{itemize}
				\item Cardiff Open Sauce Hackathon, \textit{2013 - \textbf{Award:} prize for best team}
			\end{itemize}
		\end{itemize}
	}
	
	\cvitem{Projects}{\textit{Examples of software projects completed in my spare time (more on request):}
		\begin{itemize}
			\item WekaPy - \textit{Python wrapper for the WEKA machine-learning toolkit}
			\item Gower Tides - \textit{Android (\textsc{Java}) app for tidal, surf, and weather reports for the Gower}	
			\item \textit{Various apps/sites for friends \& family}
		\end{itemize}
	}


%----------------------------------------------------------------------------------------
%	TRAINING SECTION
%----------------------------------------------------------------------------------------	

\section{Relevant training}
	\cvitem{2010}{Demonstrating Laboratory Based Teaching - \textit{UGC}}
	\cvitem{2010}{Assessing Student Learning in the Sciences - \textit{UGC}}


	
%----------------------------------------------------------------------------------------
%	AWARDS SECTION
%----------------------------------------------------------------------------------------

\section{Academic Awards}
	\cvitem{2013}{
		\begin{itemize}
			\item Department award for best research presentation
			\item Department award for best research poster 
			\item Student award for best research poster
		\end{itemize}}
	\cvitem{2012}{
		\begin{itemize}
			\item Awarded \pounds850 funding to host an interdisciplinary hackathon between SOCSI and COMSC
			\item Department award for best research presentation
		\end{itemize}}
	\cvitem{2011}{
		\begin{itemize}
			\item Student award for best research poster
		\end{itemize}}
			
			
%----------------------------------------------------------------------------------------
%	COMPUTER SKILLS SECTION
%----------------------------------------------------------------------------------------

\section{Skillset}
	\subsection{General}
    \hspace{2.5cm}\parbox[b][3em][t]{0.8\textwidth}{I have excellent written and verbal communication skills, I am experienced in working in interdisciplinary and distributed research teams and collaborations and I enjoy conveying ideas and research through reports and presentations.}

	\subsection{Technical}
		\hspace{2.5cm}\textit{I am experienced and enjoy working with the following languages and 					techniques.}\\
		\cvitem{Proficient}{Python, \textsc{Java}, PHP, BASH, SQL, \LaTeX, HTML \& CSS, JavaScript (inc. JQuery, etc.)}
		\cvitem{Intermediate}{Perl, Ruby, C, C++}

	\subsection{Platforms and Technologies}
		\hspace{2.5cm}\textit{I am also used to working with the following technologies, systems, and 				platforms.}\\
		\cvitem{Proficient}{Git, Android, GNU/Linux, UNIX, Flask, NoSQL platforms (MongoDB, CouchDB, etc.), WEKA}
		\cvitem{Intermediate}{Django, Node.js, Rails}

\end{document}
