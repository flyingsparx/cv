%----------------------------------------------------------------------------------------
%	PACKAGES AND OTHER DOCUMENT CONFIGURATIONS
%----------------------------------------------------------------------------------------

\documentclass[11pt,a4paper,sans]{moderncv} % Font sizes: 10, 11, or 12; paper sizes: a4paper, letterpaper, a5paper, legalpaper, executivepaper or landscape; font families: sans or roman

\moderncvstyle{classic} % CV theme - options include: 'casual' (default), 'classic', 'oldstyle' and 'banking'
\moderncvcolor{green} % CV color - options include: 'blue' (default), 'orange', 'green', 'red', 'purple', 'grey' and 'black'

\usepackage{lipsum} % Used for inserting dummy 'Lorem ipsum' text into the template

\usepackage[scale=0.75]{geometry} % Reduce document margins
%\setlength{\hintscolumnwidth}{3cm} % Uncomment to change the width of the dates column
%\setlength{\makecvtitlenamewidth}{10cm} % For the 'classic' style, uncomment to adjust the width of the space allocated to your name


%----------------------------------------------------------------------------------------
%	NAME AND CONTACT INFORMATION SECTION
%----------------------------------------------------------------------------------------

\firstname{Will} % Your first name
\familyname{Webberley} % Your last name

% All information in this block is optional, comment out any lines you don't need
\title{Curriculum Vitae}
\address{40 Planet St}{Cardiff, CF24 0HZ}
\mobile{07787 567865}
\email{will@flyingsparx.net}
\homepage{flyingsparx.net} % The first argument is the url for the clickable link, the second argument is the url displayed in the template - this allows special characters to be displayed such as the tilde in this example
\social[twitter]{@flyingSparx} 
%\extrainfo{additional information}
%\photo[70pt][0.4pt]{pictures/picture} % The first bracket is the picture height, the second is the thickness of the frame around the picture (0pt for no frame)
%\quote{"A witty and playful quotation" - John Smith}

%----------------------------------------------------------------------------------------

\begin{document}

\makecvtitle % Print the CV title

%----------------------------------------------------------------------------------------
%	EDUCATION SECTION
%----------------------------------------------------------------------------------------

\section{PhD Thesis}

\cvitem{Title}{\emph{Inferring Interestingness in Online Social Networks}}
\cvitem{Supervisors}{Stuart M Allen \& Roger M Whitaker}
\cvitem{Description}{Exploring information propagation in online social networks. In particular, the use of machine learning to infer interesting Tweets in Twitter.}

\section{Education}

\cventry{2010--Present}{PhD Computer Science}{Cardiff University}{UK}{}{Distributed and Scientific Computing}  % Arguments not required can be left empty
\cventry{2007--2010}{BSc Computer Science}{Cardiff University}{UK}{}{First Class Honours}
\cventry{2006--2007}{A-Levels}{King's School, Worcester}{UK}{}{
	\textit{To A2-level:} Mathematics, Biology, Chemistry \\
	\textit{To AS-level:} Physics
}

%----------------------------------------------------------------------------------------
%	EDUCATION SECTION
%----------------------------------------------------------------------------------------

\section{Publications}
\subsection{Research}
\cventry{Sept. 2013}{Inferring the Interesting Tweets in Your Network}
		{\newline Will Webberley, Stuart M Allen, Roger M Whitaker}
		{\newline in Workshop on Analyzing Social Media for the Benefit of Society (SOCIETY2.0), 3\textsuperscript{rd} International Conference on Social Computing and its Applications (SCA)}{}			
		{\textit{At:} Karlsruhe, Germany / \textit{Reviewed:} Blind peer-reviewed / \textit{Publisher:} IEEE}
	
\cventry{Sept. 2011}{Retweeting: A Study of Message-Forwarding in Twitter}
	{\newline Will Webberley, Stuart M Allen, Roger M Whitaker}
	{\newline in Workshop on Mobile and Online Social Networks (MOSN'11), 5\textsuperscript{th} International Conference on Network and System Security (NSS)}{}{\textit{At:} Milan, Italy / \textit{Reviewed:} Double-blind peer-reviewed / \textit{Publisher:} IEEE}
	
\subsection{Miscellaneous}
\cventry{Jul. 2013}{Direct-to-S3 File Uploads in Python}{\newline Will Webberley}{\newline{Heroku}}{}
	{An article written for Heroku's Dev Center to provide guidance on performing asynchronous uploads to Amazon's S3 platform using JavaScript and Python as a signing server.}
	
%----------------------------------------------------------------------------------------
%	WORK EXPERIENCE SECTION
%----------------------------------------------------------------------------------------

\section{Employment}

\cventry{2010--Present}{Tutor}{Cardiff University}{UK}{}{Practical lab and tutorial sessions for undergraduate and masters students.
\newline{}\textit{Topics include:} Python, Java, computer architecture, assembly language, mobile communications and meta heuristics, algorithms and data structures}

\cventry{2010}{Abercrombie \& Fitch}{Hollister Impact}{}{}{Responsible for organising the shop floor, receiving and oragnising shipments, keeping track of stock and replenishment. Skills gained included communication accross teams in a multi-discipline environment, precise organisation and being accurate with stock.}

\cventry{2009}{Pulse Nightclub}{Bar staff}{}{}{Bar work as part of a vibrant and outgoing team. Direct contact with customers helped improve communication and teamwork skills, confidence, quick-thinking in a very fast-paced environment and over prolonged lengths of time. Responsible also for the supervision of new members of staff.}



%----------------------------------------------------------------------------------------
%	AWARDS SECTION
%----------------------------------------------------------------------------------------

\section{Awards}

\cvitem{2013}{
		\begin{itemize}
			\item Department award for best research presentation
			\item Department award for best research poster 
			\item Student award for best research poster
		\end{itemize}}
\cvitem{2012}{
		\begin{itemize}
			\item Awarded \pounds850 funding to host an interdisciplinary hackathon between computer- and social-scientists
			\item Department award for best research presentation
		\end{itemize}}
\cvitem{2011}{
		\begin{itemize}
			\item Student award for best research poster
		\end{itemize}}
		
%----------------------------------------------------------------------------------------
%	COMPUTER SKILLS SECTION
%----------------------------------------------------------------------------------------

\section{Skillset}
\subsection{Technical}
\hspace{2.5cm}\textit{I am experienced and enjoy working with the following languages and techniques.}\\
\cvitem{Proficient}{Python, \textsc{java}, C / C++, BASH, SQL, \LaTeX}
\cvitem{Intermediate}{Perl, Ruby}

\subsection{Technologies}
\hspace{2.5cm}\textit{I have taken part in projects using the following technologies and platforms.}\\
\cvitem{Proficient}{Flask, MongoDB, CouchDB}
\cvitem{Intermediate}{Django, node.js}

\section{Teaching Experience}
As part of my PhD, I've taught in the School of Computer Science and Informatics at Cardiff University. Teaching enabled me to gain experiences and skills in speaking and instructing in front of large groups of people, providing a positive atmosphere for others to work in and in giving construcitve feedback.

Assessments have allowed me to gain insights into how other people understand how different processes work, and how people can sometimes understand things slightly differently.

Areas I've taught in include Java, Python, Assembly Language, system design (UML, etc.), project management, mobile communications, business processes, algorithmic programming, important professional skills, and so on. I've also had an opportunity to supervise serveral year-long projects carried out by teams of around 6-10 people, enabling me to gain more skills in project management, and more well-rounded experiences all round.

%----------------------------------------------------------------------------------------
%	INTERESTS SECTION
%----------------------------------------------------------------------------------------

\section{Interests}

\renewcommand{\listitemsymbol}{-~} % Changes the symbol used for lists

\cvlistdoubleitem{Piano}{Chess}
\cvlistdoubleitem{Cooking}{Dancing}
\cvlistitem{Running}

\end{document}