%----------------------------------------------------------------------------------------
%	PACKAGES AND OTHER DOCUMENT CONFIGURATIONS
%----------------------------------------------------------------------------------------

\documentclass[11pt,a4paper,sans]{moderncv} % Font sizes: 10, 11, or 12; paper sizes: a4paper, letterpaper, a5paper, legalpaper, executivepaper or landscape; font families: sans or roman

\moderncvstyle{classic} % CV theme - options include: 'casual' (default), 'classic', 'oldstyle' and 'banking'
\moderncvcolor{orange} % CV color - options include: 'blue' (default), 'orange', 'green', 'red', 'purple', 'grey' and 'black'

\usepackage[scale=0.85]{geometry} % Reduce document margins
\usepackage{color}
%\setlength{\hintscolumnwidth}{3cm} % Uncomment to change the width of the dates column
\setlength{\makecvtitlenamewidth}{10cm} % For the 'classic' style, uncomment to adjust the width of the space allocated to your name
%\setlength{\makecvtitlenameheight}{5cm}

%----------------------------------------------------------------------------------------
%	NAME AND CONTACT INFORMATION SECTION
%----------------------------------------------------------------------------------------

\firstname{Will} % Your first name
\familyname{Webberley} % Your last name

% All information in this block is optional, comment out any lines you don't need
\title{\small{Data, software systems, development, and the web}}
\address{40 Planet Street, Cardiff, CF24 0HZ}
\mobile{07787 567865}
\email{will@flyingsparx.net}
\homepage{flyingsparx.net}
\social[linkedin]{flyingsparx}
\social[twitter]{flyingsparx}
\social[github]{flyingsparx} 


%----------------------------------------------------------------------------------------

\begin{document}
\makecvtitle

\vspace{-1cm}
%----------------------------------------------------------------------------------------
%	SKILLS SECTION
%----------------------------------------------------------------------------------------

\parbox[b][5em][t]{1.0\textwidth}{I am extremely passionate in the field of technology and development with a natural flair for creating quality systems for a wide range of uses and users. I'm always excited to learn about new technologies from others and through self-taught means. I am a skilled communicator with expertise in working in distributed and multi-disciplinary teams and I enjoy conveying thoughts and ideas through presentations, talks, and discussions.}

\section{Skillset}
\textbf{I'm highly proficient in and frequently use a wide range of languages, frameworks, and platforms.}
	\subsection{Languages}
		\cvitem{Proficient}{Python, \textsc{Java}, PHP, Go, JavaScript, Shell, SQL, \LaTeX, HTML \& CSS}
		\cvitem{Intermediate}{Perl, Ruby, C, C++}

	\subsection{Platforms, Technologies, Techniques, and Frameworks}
        \cvitem{Proficient}{Git, Android dev., GNU/Linux \& UNIX-like systems, Flask, Meteor \& Node.js, MongoDB \& CouchDB, jQuery \& AngularJS, WebSocket, Weka, Heroku, AWS, Nginx, Vim, Eclipse/NetBeans}
		\cvitem{Intermediate}{Django, Rails, iOS dev.}
        {\color{gray} \rule{\linewidth}{0.1mm} }
        
        \parbox[b][2em][t]{\textwidth}{I am very experienced with and have expertise in writing and consuming various web-based APIs, and working in object-oriented and test-driven environments for platforms ranging from distributed servers to desktop and mobile.}

	
%----------------------------------------------------------------------------------------
%	WORK EXPERIENCE SECTION
%----------------------------------------------------------------------------------------

\section{Recent Experience \& Employment}
      \cventry{2015--2017}{Simply Do Ideas}{Chief Technology Officer}{Cardiff, UK}{}{
      Technical lead on and responsibility for company products.}
    \cventry{2015}{Chaser Technologies}{Software Engineer}{London, UK}{}{
        Building and maintaining Chaser systems, databases, APIs, and apps. I also led the data analysis processes managing large financial datasets.
    }
    \cventry{2013--Present}{Cardiff University}{Research Associate and Lecturer}{Cardiff, UK}{}{
        \textbf{Research} as part of the International Technology Alliance (ITA) coalition between the UK MoD and the US Army Research Labs. I worked with a variety of US and UK academic and industry partners, such as IBM and Boeing, in advancing information and network sciences.
       \newline{}\textbf{Lecturer} of postgraduate masters (MSc Advanced Computer Science) students as module leader of Web and Social Computing, and two Computer Science undergraduate courses - Human-Computer Interaction (as module leader of \textasciitilde140 students), and Algorithms \& Data Structures.
       \newline I received consistently good feedback from peers and students, both in terms of technical knowledge and in style \& approach. I was responsible for developing course materials, syllabae, exams, and coursework.
    }

	\cventry{2010--2013}{Cardiff University}{Tutor and student supervisor}{Cardiff, UK}{}{
        Tutoring, supervision, and assessment of BSc and MSc students in practical and theory-based sessions covering a variety of topics ranging from theory, mobile systems, heuristics, algorithms, and general programming.
        \newline{}\textbf{Examples of projects supervised:} Social media visualisation toolkits, digital asset stores.
    }


%----------------------------------------------------------------------------------------
%	EDUCATION SECTION
%----------------------------------------------------------------------------------------
\newpage
\section{Education}
	\cventry{2010--2014}{PhD Computer Science}{Cardiff University}{}{\newline{}"\textit{Inferring Interestingness in Online Social Networks}"}{Exploring information propagation in online social networks. In particular, the use of Bayesian machine learning methodologies in the prediction of tweet popularity in Twitter as a basis for the inference and ranking of interesting user-contributed data.\\
            I have periodically published interesting findings and analyses in proceedings, and have contributed to the ``excellent''-rated EU Recognition project.\\
		}
	\cventry{2007--2010}{BSc Computer Science}{Cardiff University}{First-class honours}{}{}


%----------------------------------------------------------------------------------------
%	COMMUNITY SECTION
%----------------------------------------------------------------------------------------


\section{Community Involvement, Contribution \& Output}
    \cvitem{Projects}{\textit{Examples of projects worked-on in spare time or in freelance:}
		\begin{itemize}
            \item \textbf{WekaPy} \& \textbf{WekaGo} - \textit{Python and Go wrappers for the Weka machine-learning toolkit}
      \item \textbf{Gower Tides} - \textit{Native Android \& iOS apps showing tidal events \& surf/weather forecasts}
            \item \textbf{CENode} - \textit{Support human-machine and machine-machine communication at the network edge, working with Cardiff University and IBM UK}
		\end{itemize}
        For more, please see \underline{github.com/flyingsparx} and \underline{flyingsparx.net/project}.
	}

    \cvitem{First-author peer-reviewed publications}{
        \begin{itemize}
            \item "Retweeting beyond Expectation: Inferring Interestingness in Twitter", \textit{W.M. Webberley, S.M. Allen, R.M. Whitaker}, under review for the journal of \textit{Computer Communications: Special Issue on Online Social Networks}. 2015.
            \item "Inferring the Interesting Tweets in Your Network", \textit{W.M. Webberley, S.M. Allen, R.M. Whitaker}, in \textit{Third International Conference on Cloud and Green Computing}. Karsruhe, Germany, 2013.
            \item "Retweeting: A Study of Message-Forwarding in Twitter", \textit{W.M. Webberley, S.M. Allen, R.M. Whitaker}, in \textit{Workshop on Mobile and Online Social Networks, 2011}. Milan, Italy, 2011.
        \end{itemize}
        For more, please see \underline{scholar.google.com/citations?user=b2RThHcAAAAJ}.
    }

	\cvitem{Written}{
		\begin{itemize}
            \item Invited articles for the Heroku Dev Center:
            \begin{itemize}
                \item "Direct-to-S3 File Uploads in Node.js", \textit{Heroku 2014}
                \item "Direct-to-S3 File Uploads in Python", \textit{Heroku 2013}
			\end{itemize}
            \item I aso maintain an active blog, which is hosted on my website at \underline{flyingsparx.net/blog}.
		\end{itemize}
	}
	
	\cvitem{Spoken}{
		\begin{itemize}
			\item \textbf{Examples of informal talks given}\begin{itemize}		
                \item "Writing Useful Web APIs", \textit{2015}
                \item "Contributing to the Open-Source World", \textit{2014}
				\item "The Flask Microframework", \textit{2013}
				\item "Node.js: An Introduction to Development", \textit{2012}
				\item "Typesetting with \LaTeX", \textit{2012}
				\item "Developing Google Chrome Extensions", \textit{2011}
			\end{itemize}
			\item \textbf{Examples of formal and academic talks and seminars given}\begin{itemize}
                \item "Inferring Interestingness in OSNs" \textit{Guest speaker at King's College, London, 2014}
				\item "Inferring the Interesting Tweets in Your Network", \textit{SOCIETY2.0, Karlsruhe, Germany, 2013}
				\item "Finishing a PhD in Under 4 Years", \textit{Cardiff University Research Retreat, 2013}
                \item "Weka - the Machine-Learning Toolkit", \textit{Cardiff University MobiSoc Research Group, 2013}
				\item "Retweeting: A Study of Message-Forwarding in Twitter", \textit{MOSN'11, Milan, Italy, 2011}
			\end{itemize}
		\end{itemize}
	}

	\cvitem{Events}{
    I have taken part in and helped organised several community events, such as talks, seminar series, and hackathons.	
	}


\textbf{Notes}\\
A digital version of my CV is available in colour at \underline{flyingsparx.net/cv}.\\
The source code for my CV is available at \underline{github.com/flyingsparx/cv}.


\end{document}
