\documentclass[11pt,a4paper]{article}
\usepackage[utf8]{inputenc}
\usepackage{amsmath}
\usepackage{amsfonts}
\usepackage{amssymb}
\usepackage[margin=0.87in]{geometry}
\title{William M. Webberley - Supporting Information}
\date{}
\begin{document}
\maketitle
Please find information below to support my claims to the position's required essential criteria.\\

\textbf{Research in Social Media \& Situational Awareness}\\
Many aspects of my PhD research is based around social media extraction and analysis for the purposes of understanding the relationships between user behaviour and their relative positions on the social graph. Much of this work has been published in international conference proceedings (including Italy and Germany).\\
Understanding social situation awareness has been an important factor during my time working on the Sentinel project, which is centred around the union of locational (situational) and semantic social information.

\textbf{Programming Experience}\\
I am a very enthusiastic programmer, both in research and in my spare time. I enjoy working with an array of languages and on a variety of platforms. I am most experienced with JavaScript (both client- and server-side), as evidenced by my research work with Sentinel and other work, including web-apps and device-optimised sites, published on GitHub and elsewhere. I also maintain two articles for Heroku's Dev Center, focussed largely on JavaScript.    

\textbf{Evidence of Oral Communication Skills}\\
I have demonstrated excellent verbal skills through presentation-style tutorials and lecture-type sessions to students, informal talks, and formal research talks in domestic events and international conference workshops. Over the past few years I have been awarded the School of Computer Science \& Informatics' ``best presentation'' award for research students on multiple occasions.\\
During my PhD I have been able to demonstrate my communication skills with research peers at different levels, such as my supervisors, fellow research students, my research group (and `MobiSoc'), and with non-native English-speaking researchers at overseas conferences, including social scientists at a recent SCA conference. I have been able to show my ability to communicate ith less-technical collaborators, such as social scientists, during my time on the Sentinel project, since this project is largely aimed towards social-oriented problems.

\textbf{Quantitive and Qualitative Research Skills}\\
I am aware of and have experience with various quantitative research techniques, from data collection and storage, through to the analysis of the data for driving statistical outcomes. I enjoy discovering new features in data in order to support findings by providing further rigidity to quantitative results. I am experienced with Python for the task of data analysis, as well as statistical tools, such as R and Matlab, and crowdsourcing services, such as Mechanical Turk and Crowdflowwer.\\
The initial parts of my PhD and other research also illustrates my ability to carry out qualitative research. This has included researching relevant literature, discovering problems with information-sharing on social platforms, and using findings with eht objective of shaping the direction and goals of quantitative research. 

\textbf{Communicating Written Specialist Ideas}\\
The field of computer science is among those that sometimes require understanding hypothetically conceptual (and non-practical) situations, and when teaching these ideas to less-experienced undergraduate students alongside my PhD, I have been able to show my ability to convey the information in an effective and comprehensive fashion, both on paper and orally.\\
Throughout my PhD and undergraduate degree I have produced many instances of high quality written work, including the publication of academic research papers in international proceedings and funding proposals. The intended audience of much of my written work is varied, so it is important that I am able to convey specialist concepts clearly to both technical and non-technical audiences.

\textbf{Developing Research Methods for Social-Driven Decision-Support}\\
Most of my research to date has been focussed on social media analysis, in terms of the social graphs and the behavious and relationships between the users, providing me with excellent experience and knowledge in data capture and the analysis of social data.\\
More specifically, my PhD research has used social media analysis as a basis for identifying useful and interesting user-contributed information, allowing exceptional information to be flagged up to be actioned upon. This work in particular is very relevant to social-driven decision-support. 

\textbf{Experience with Software Engineering Processes in App \& API Development}\\
I am very experienced with the processes behind HTML mobile app development, as evidenced with work on the Sentinel project and other efforts in my research and  spare time. In addition, I am competent with Android development and have some experience with iOS.\\
Many of these apps required the creation of APIs, as well as the consumption of external ones. I have proven experience with many external services, for social and otherwise, including Twitter, Facebook, Google, and environmental APIs. I observe peer-reviewed and official guidelines to ensure good practice is adhered to in all app development.  

\textbf{Evidence of Organisational Skills}\\
Throughout my PhD I have been able to demonstrate strong independent research skills, which has required me to be organised in setting (and realising) goals and research plans. It has been important that I am able to structure my work to follow plans in accordance with other collaborators and the scope of the project.\\
Whilst working on the Sentinel project, I have demonstrated that I am able to follow an agreed schedule (where possible) and be organised in delivering intermediate work throughout. 

\textbf{Evidence of Working to Timeframes}\\
When planning research papers and external talks, it has been important that I illustrate my ability to organise and carry out the work so that I am able to complete research, data collection, analyses, and evaluations within the required deadline, and to ensure that the submitted work is of a high quality.

\textbf{Higher-Level Degree Status}\\
Having submitted a first draft of my thesis for review by my supervisors in January 2014, I am now in the final stages of making ammendments and corrections with the hope of submitting it in April.

\end{document}
